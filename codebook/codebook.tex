\documentclass[a4paper,10pt,twocolumn,oneside]{article}
\setlength{\columnsep}{10pt}                                                                    %兩欄模式的間距
\setlength{\columnseprule}{0pt}                                                                %兩欄模式間格線粗細

\usepackage{amsthm}								%定義,例題
\usepackage{amssymb}
%\usepackage[margin=2cm]{geometry}
\usepackage{fontspec}								%設定字體
\usepackage{color}
\usepackage[x11names]{xcolor}
\usepackage{listings}								%顯示code用的
%\usepackage[Glenn]{fncychap}						%排版,頁面模板
\usepackage{fancyhdr}								%設定頁首頁尾
\usepackage{graphicx}								%Graphic
\usepackage{enumerate}
\usepackage{titlesec}
\usepackage{amsmath}
\usepackage[CheckSingle, CJKmath]{xeCJK}
% \usepackage{CJKulem}

%\usepackage[T1]{fontenc}
\usepackage{amsmath, courier, listings, fancyhdr, graphicx}
\topmargin=0pt
\headsep=5pt
\textheight=780pt
\footskip=0pt
\voffset=-40pt
\textwidth=545pt
\marginparsep=0pt
\marginparwidth=0pt
\marginparpush=0pt
\oddsidemargin=0pt
\evensidemargin=0pt
\hoffset=-42pt

\titlespacing\subsection{0pt}{4pt plus 2pt minus 2pt}{0pt plus 2pt minus 2pt}


%\renewcommand\listfigurename{圖目錄}
%\renewcommand\listtablename{表目錄} 

%%%%%%%%%%%%%%%%%%%%%%%%%%%%%

\setmainfont{Consolas}				%主要字型
%\setmonofont{Monaco}				%主要字型
\setmonofont{Consolas}
\setCJKmainfont{Noto Sans CJK TC}
% \setCJKmainfont{Consolas}			%中文字型
%\setmainfont{sourcecodepro}
\XeTeXlinebreaklocale "zh"						%中文自動換行
\XeTeXlinebreakskip = 0pt plus 1pt				%設定段落之間的距離
\setcounter{secnumdepth}{3}						%目錄顯示第三層

%%%%%%%%%%%%%%%%%%%%%%%%%%%%%
\newcommand\digitstyle{\color{DarkOrchid3}}
\makeatletter
\lst@CCPutMacro\lst@ProcessOther {"2D}{\lst@ttfamily{-{}}{-{}}}
\@empty\z@\@empty

\newtoks\BBQube@token
\newcount\BBQube@length
\def\BBQube@ResetToken{\BBQube@token{}\BBQube@length\z@}
\def\BBQube@Append#1{\advance\BBQube@length\@ne
  \BBQube@token=\expandafter{\the\BBQube@token#1}}

\def\BBQube@ProcessChar#1{%
  \ifnum\lst@mode=\lst@Pmode%
    \ifnum 9<1#1%
      \expandafter\BBQube@Append{\begingroup\digitstyle #1 \endgroup}%
    \else%
      \expandafter\BBQube@Append{#1}%
    \fi%
  \else%
    \expandafter\BBQube@Append{#1}%
  \fi%
}
\def\BBQube@ProcessStringInner#1#2\BBQube@nil{%
  \expandafter\BBQube@ProcessChar{#1}%
  \if\relax\detokenize{#2}\relax%
  \else%
    \expandafter\BBQube@ProcessStringInner#2\BBQube@nil%
  \fi%
}

\def\BBQube@ProcessString#1{\expandafter\BBQube@ProcessStringInner#1\BBQube@nil}

\lst@AddToHook{OutputOther}{%
\BBQube@ResetToken%
\expandafter\BBQube@ProcessString{\the\lst@token}%
\lst@token=\expandafter{\the\BBQube@token}%
}
\makeatother
\lstset{											% Code顯示
language=C++,										% the language of the code
basicstyle=\footnotesize\ttfamily, 						% the size of the fonts that are used for the code
%numbers=left,										% where to put the line-numbers
numberstyle=\footnotesize,						% the size of the fonts that are used for the line-numbers
stepnumber=1,										% the step between two line-numbers. If it's 1, each line  will be numbered
numbersep=5pt,										% how far the line-numbers are from the code
backgroundcolor=\color{white},					% choose the background color. You must add \usepackage{color}
showspaces=false,									% show spaces adding particular underscores
showstringspaces=false,							% underline spaces within strings
showtabs=false,									% show tabs within strings adding particular underscores
frame=false,											% adds a frame around the code
tabsize=2,											% sets default tabsize to 2 spaces
captionpos=b,										% sets the caption-position to bottom
breaklines=true,									% sets automatic line breaking
breakatwhitespace=false,							% sets if automatic breaks should only happen at whitespace
escapeinside={\%*}{*)},							% if you want to add a comment within your code
morekeywords={constexpr},									% if you want to add more keywords to the set
keywordstyle=\bfseries\color{Blue1},
commentstyle=\itshape\color{Red4},
stringstyle=\itshape\color{Green4},
}

%%%%%%%%%%%%%%%%%%%%%%%%%%%%%

\begin{document}
\pagestyle{fancy}
\fancyfoot{}
%\fancyfoot[R]{\includegraphics[width=20pt]{ironwood.jpg}}
\fancyhead[L]{National Taiwan University 8BQube}
\fancyhead[R]{\thepage}
\renewcommand{\headrulewidth}{0.4pt}
\renewcommand{\contentsname}{Contents} 

\scriptsize
\tableofcontents
%%%%%%%%%%%%%%%%%%%%%%%%%%%%%

%\newpage

\section{Basic}
\subsection{Shell script}
\lstinputlisting{1_Basic/Shell_script.cpp}
\subsection{Default code}
\lstinputlisting{1_Basic/Default_code.cpp}
\subsection{vimrc}
\lstinputlisting{1_Basic/vimrc.cpp}
\subsection{readchar}
\lstinputlisting{1_Basic/readchar.cpp}
\subsection{Black Magic}
\lstinputlisting{1_Basic/black_magic.cpp}
\subsection{Texas hold'em}
\lstinputlisting{1_Basic/Texas_holdem.cpp}


\section{Graph}
\subsection{BCC Vertex}
\lstinputlisting{2_Graph/BCC_Vertex.cpp}
\subsection{Bridge}
\lstinputlisting{2_Graph/Bridge.cpp}
\subsection{Strongly Connected Components}
\lstinputlisting{2_Graph/Strongly_Connected_Components.cpp}
\subsection{MinimumMeanCycle}
\lstinputlisting{2_Graph/MinimumMeanCycle.cpp}
\subsection{Virtual Tree}
\lstinputlisting{2_Graph/Virtual_Tree.cpp}
\subsection{Maximum Clique}
\lstinputlisting{2_Graph/Maximum_Clique.cpp}
\subsection{MinimumSteinerTree}
\lstinputlisting{2_Graph/MinimumSteinerTree.cpp}
\subsection{Dominator Tree}
\lstinputlisting{2_Graph/Dominator_Tree.cpp}
\subsection{Minimum Arborescence}
\lstinputlisting{2_Graph/Minimum_Arborescence.cpp}
\subsection{Theory}
\lstinputlisting{2_Graph/Theory.cpp}


\section{Data Structure}
\subsection{Treap}
\lstinputlisting{3_Data_Structure/Treap.cpp}
\subsection{Leftist Tree}
\lstinputlisting{3_Data_Structure/Leftist_Tree.cpp}
\subsection{Heavy light Decomposition}
\lstinputlisting{3_Data_Structure/Heavy_light_Decomposition.cpp}
\subsection{Smart Pointer}
\lstinputlisting{3_Data_Structure/Smart_Pointer.cpp}
\subsection{LiChaoST}
\lstinputlisting{3_Data_Structure/LiChaoST.cpp}
\subsection{link cut tree}
\lstinputlisting{3_Data_Structure/link_cut_tree.cpp}
\subsection{KDTree}
\lstinputlisting{3_Data_Structure/KDTree_useful.cpp}


\section{Flow/Matching}
\subsection{Dinic}
\lstinputlisting{4_Flow_Matching/Dinic.cpp}
\subsection{Kuhn Munkres}
\lstinputlisting{4_Flow_Matching/Kuhn_Munkres.cpp}
\subsection{MincostMaxflow}
\lstinputlisting{4_Flow_Matching/MincostMaxflow.cpp}
\subsection{Maximum Simple Graph Matching}
\lstinputlisting{4_Flow_Matching/Maximum_Simple_Graph_Matching.cpp}
\subsection{Minimum Weight Matching (Clique version)}
\lstinputlisting{4_Flow_Matching/Minimum_Weight_Matching.cpp}
\subsection{SW-mincut}
\lstinputlisting{4_Flow_Matching/SW-mincut.cpp}
\subsection{BoundedFlow}
\lstinputlisting{4_Flow_Matching/BoundedFlow.cpp}
\subsection{Gomory Hu tree}
\lstinputlisting{4_Flow_Matching/Gomory_Hu_tree.cpp}
\subsection{NumberofMaximalClique}
\lstinputlisting{4_Flow_Matching/NumberofMaximalClique.cpp}
\subsection{isap}
\lstinputlisting{4_Flow_Matching/isap.cpp}


\section{String}
\subsection{KMP}
\lstinputlisting{5_String/KMP.cpp}
\subsection{Z-value}
\lstinputlisting{5_String/Z-value.cpp}
\subsection{Manacher}
\lstinputlisting{5_String/Manacher.cpp}
\subsection{Suffix Array}
\lstinputlisting{5_String/Suffix_Array.cpp}
\subsection{SAIS}
\lstinputlisting{5_String/SAIS.cpp}
\subsection{Aho-Corasick Automatan}
\lstinputlisting{5_String/Aho-Corasick_Automatan.cpp}
\subsection{Smallest Rotation}
\lstinputlisting{5_String/Smallest_Rotation.cpp}
\lstinputlisting{5_String/Manacher.cpp}
\subsection{De Bruijn sequence}
\lstinputlisting{5_String/De_Bruijn_sequence.cpp}
\subsection{SAM}
\lstinputlisting{5_String/SAM.cpp}
\subsection{PalTree}
\lstinputlisting{5_String/PalTree.cpp}
\subsection{cyclicLCS}
\lstinputlisting{5_String/cyclicLCS.cpp}


\section{Math}
\subsection{ax+by=gcd}
\lstinputlisting{6_Math/ax+by=gcd.cpp}
\subsection{floor and ceil}
\lstinputlisting{6_Math/floor_ceil.cpp}
\subsection{Miller Rabin}
\lstinputlisting{6_Math/Miller_Rabin.cpp}
\subsection{Big number}
\lstinputlisting{6_Math/Big_number.cpp}
\subsection{Fraction}
\lstinputlisting{6_Math/Fraction.cpp}
\subsection{Simultaneous Equations}
\lstinputlisting{6_Math/Simultaneous_Equations.cpp}
\subsection{Pollard Rho}
\lstinputlisting{6_Math/Pollard_Rho.cpp}
\subsection{Simplex Algorithm}
\lstinputlisting{6_Math/Simplex_Algorithm.cpp}
\subsection{chineseRemainder}
\lstinputlisting{6_Math/chineseRemainder.cpp}
\subsection{QuadraticResidue}
\lstinputlisting{6_Math/QuadraticResidue.cpp}
\subsection{PiCount}
\lstinputlisting{6_Math/PiCount.cpp}
\subsection{Algorithms about Primes}
\lstinputlisting{6_Math/Algorithms_about_Primes.cpp}


\section{Polynomial}
\subsection{Fast Fourier Transform}
\lstinputlisting{7_Polynomial/Fast_Fourier_Transform.cpp}
\subsection{Number Theory Transform}
\lstinputlisting{7_Polynomial/Number_Theory_Transform.cpp}
\subsection{Fast Walsh Transform}
\lstinputlisting{7_Polynomial/Fast_Walsh_Transform.cpp}
\subsection{Polynomial Operation}
\lstinputlisting{7_Polynomial/Polynomial_Operation.cpp}


\section{Geometry}
\subsection{Default Code}
\lstinputlisting{8_Geometry/Default_code.cpp}
\subsection{Convex hull}
\lstinputlisting{8_Geometry/Convex_hull.cpp}
\subsection{External bisector}
\lstinputlisting{8_Geometry/external_bisector.cpp}
\subsection{Heart}
\lstinputlisting{8_Geometry/Heart.cpp}
\subsection{Polar Angle Sort}
\lstinputlisting{8_Geometry/Polar_Angle_Sort.cpp}
\subsection{Intersection of two circles}
\lstinputlisting{8_Geometry/Intersection_of_two_circles.cpp}
\subsection{Intersection of polygon and circle}
\lstinputlisting{8_Geometry/Intersection_of_polygon_and_circle.cpp}
\subsection{Intersection of line and circle}
\lstinputlisting{8_Geometry/Intersection_of_line_and_circle.cpp}
\subsection{Half plane intersection}
\lstinputlisting{8_Geometry/Half_plane_intersection.cpp}
\subsection{Convexhull3D}
\lstinputlisting{8_Geometry/Convexhull3D.cpp}
\subsection{CircleCover}
\lstinputlisting{8_Geometry/CircleCover.cpp}
\subsection{DelaunayTriangulation}
\lstinputlisting{8_Geometry/DelaunayTriangulation.cpp}
\subsection{Tangent line of two circles}
\lstinputlisting{8_Geometry/Tangent_line_of_two_circles.cpp}
\subsection{minMaxEnclosingRectangle}
\lstinputlisting{8_Geometry/minMaxEnclosingRectangle.cpp}
\subsection{Minkowski Sum}
\lstinputlisting{8_Geometry/Minkowski_Sum.cpp}


\section{Else}
\subsection{Mo's Alogrithm(With modification)}
\lstinputlisting{9_Else/Mos_Alogrithm_With_modification.cpp}
\subsection{Mo's Alogrithm On Tree}
\lstinputlisting{9_Else/Mos_Alogrithm_On_Tree.cpp}
\subsection{DynamicConvexTrick}
\lstinputlisting{9_Else/DynamicConvexTrick.cpp}


\section{JAVA}
\subsection{Big number}
\lstinputlisting{10_JAVA/Big_number.cpp}



\end{document}
